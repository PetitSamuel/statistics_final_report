\documentclass[14.5pt]{article}
\usepackage{fancyhdr}
\usepackage{amsmath}
\pagestyle{fancy}

\lhead{Samuel Petit 17333946 petits@tcd.ie}
\rhead{Statistical Methods Final Exam}
\renewcommand{\headrulewidth}{0.4pt}
\renewcommand{\footrulewidth}{0.4pt}

\begin{document}

\section*{Question 1}
\subsection*{Part a}

We have a total of 10 topics and need to pick 3, thus the amount of possible combinations
is $\begin{pmatrix} 10 \\ 3 \end{pmatrix} = 120$.

\subsection*{Part b}

To find an expression for the probability that none of the n topics studied come up,
I first find an expression for the opposite. That is at least 1 of the topics studied
come up in the exam.
To find that expression, I use the amount of combinations possible for 3 topics drawn out
of 10. I then need to find the amount of combinations such that one or more questions out
of n studied come up. This comes down to: $\begin{pmatrix} n \\ 3 \end{pmatrix}$.
Thus, the probability that one or more questions studied come up in the exam is:
\begin{equation}
    \frac{\begin{pmatrix} n \\ 3 \end{pmatrix}}{\begin{pmatrix} 10 \\ 3 \end{pmatrix}}
\end{equation}
We are looking for the opposite thus the probability that none of the n studied topics come up is:
\begin{equation}
    1 - \frac{\begin{pmatrix} n \\ 3 \end{pmatrix}}{\begin{pmatrix} 10 \\ 3 \end{pmatrix}}
\end{equation}
\subsection*{Part c}
\subsection*{Part d}
\subsection*{Part e}
\subsection*{Part f}
\subsection*{Part g}


\end{document}